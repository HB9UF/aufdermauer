\documentclass[a4paper,11pt,halfparskip,smallheadings,DIV=10]{scrartcl}
\usepackage{german}
\usepackage{scrpage2}
\usepackage{graphicx}
\usepackage{float}
\usepackage[pdfauthor={Mathias Weyland, HB9FRV},pdftitle={},pdfstartview=FitH,pdfborder={0 0 0}]{hyperref}
\usepackage{booktabs}
\setkomafont{pagehead}{\normalfont\normalcolor}
%\addtolength{\textheight}{5mm}
\title{HB9UF Hochfrequenz-Spürer}
\author{Mathias Weyland, HB9FRV}
\date{Version 1.1}
\pagestyle{scrheadings}
\lohead{}
\rohead{}
\lofoot{}
\cfoot{\thepage}
\rofoot{}
%\usepackage{eso-pic}
%\AddToShipoutPicture{\resizebox{0.9\pdfpagewidth}{0.9\pdfpageheight}%           
%{\rotatebox{60}{\color[gray]{0.8}\hspace*{5mm}\textsc{DRAFT}}}}
\setkomafont{pagehead}{\sf}
\sloppy
\begin{document}
\maketitle
\vspace{-1.5cm}

\section{Einführung}
Im Folgenden ist der Aufbau des HB9UF HF-Spürer kurz beschrieben -- ein
Gerät, welches hochfrequente Felder von Sendern oder Störungen anzeigt.
Es handelt sich dabei um ein Bauprojekt, das anlässlich des Hamfestes 2019 in
Zug vorgestellt worden ist. Ziel war es, interessierten Besuchern am Stand von
HB9UF aufzuzeigen, wie Bauprojekte mit modernen Methoden umgesetzt werden
können. Hierzu gehört neben dem Entwurf der Schaltung und dem Platinen-Layout
am PC auch das Bestellen der Bauteile, der Platine und nicht zuletzt das
Löten der Platine. Letzteres konnten die Besucher an besagtem Stand selbst
ausprobieren und einen Bausatz mit nach Hause nehmen. Die verwendeten
SMD-Bauteile sind verhältnismässig gross, um in dieser Hinsicht den
Einstieg zu erleichtern. Das Projekt wurde mit der Open-Source-Software
KiCad (Version 5.1.2) erstellt und ist unter einer freien Lizenez
(CC BY-SA 4.0) online  verfügbar\footnote{\url{https://github.com/HB9UF/aufdermauer}}.

\section{Schaltung}
Der Hochfrequenzspürer arbeitet im gesamten HF und VHF-Bereich bis
in den UHF-Bereich. Die Schaltung ist inspiriert von einem 
Artikel, der 1993 in der CQ DL erschienen ist\footnote{Einfacher
``Hochfrequenzschnüffler'' von Steve Ortmayer, G4RAW. cq-DL 1/93.}. In der
ursprünglichen Schaltung kommt für den Gleichrichter eine OA90 Germanium-Diode
zum Einsatz; der JFET ist ein 2N3819. Beide Bauteile sind heute nur noch schwer
zu beschaffen -- ein Problem das nicht selten auftritt. Diese Schaltung bietet
sich deshalb für eine Modernisierung mit SMD-Bauteilen geradezu an. Abbildung
\ref{fig:schematic} zeigt den überarbeiteten Schaltplan. Die ursprüngliche
Schaltung ist für den Einsatz mit einem Zeiger-Instrument konzipiert. Diese
Anzeige haben wir der Einfachheit halber durch eine LED ersetzt (R1, R2, Q2, D2).
Wer statt der LED weiterhin ein analoges oder digitales Messgerät nutzen
möchte, kann Q2, R2 und D2 nicht bestücken und den Spannungsabfall über
R1 messen. Es ist auch möglich, R1 nicht zu bestücken und stattdessen
den Drain-Strom von Q1 zu messen. Für beide Anwendungsfälle ist eine zweipolige JST XH
Buchse J2 vorgesehen. Die 1~mH Drossel am Antenneneingang erfährt bei höheren
Kurzwellenbändern Eigenresonanz. Die Drosselwirkung nimmt dann mit zunehmender
Frequenz ab. Die in Serie geschaltete Ferritperle L3 vermag daran nicht viel
zu ändern, zumal ihre Reaktanz verhältnismässig klein ist. Hier besteht deshalb
Optimierungspotential: Es kann mit verschiedenen Konfigurationen experimentiert
werden, wofür die Bauteile L2 und L4 vorgesehen sind. Diese sind in der
normalen Ausführung nicht bestückt. Eine weitere Bestückungsalternative ist
ein Kurzschluss über L4 und L1 bis L3 nicht einzusetzen. Der Spürer reagiert dann sehr
empfindlich auf niederfrequente Felder.

\begin{figure}
\includegraphics[width=\textwidth]{schaltung.pdf}
\caption{Schaltschema des HF-Spürers}
\label{fig:schematic}
\end{figure}

\section{Aufbau}
Der Aufbau erfolgt gemäss Abbildung \ref{fig:schematic}. Als Antenne dient ein
Draht von 30 -- 100~cm Länge, welcher an J3 gelötet wird (Beschriftung ``ANT''
auf Platinenrückseite; Bohrung im Zentrum verwenden, es kann zum
Experimentieren auch eine SMA-Buchse eingelötet werden). Ein ebensolanges
Gegengewicht ist notwendig, welches mit der Masse verbunden wird (Bohrung mit
Massensymbol unterhalb des Logos). Die Schaltung benötigt eine Spannung von 9~V und wird
über Pins 1 und 2 von J1, z.B. von einer 9~V Blockbatterie gespiesen. Pins 3
und 4 dienen dem Ein- und Ausschalten; an sie kann ein Schalter angeschlossen
werden, oder aber die beiden Pins werden mit einem Jumper einfach
kurzgeschlossen. Die 3~mm LED kann, je nach Einsatz, auf der Vorder- oder der
Rückseite eingesetzt werden.  Abbildung \ref{fig:aufbau} zeigt eine fertig
bestückte Platine.

Beim Einsatz eines Netzteiles ist zu berücksichtigen, dass das Spiesekabel
ebenfalls als Antenne wirkt. Ausserdem darf ein solches Netzteil selbst keine
Hochfrequenz produzieren, da der HF-Spürer diese unweigerlich anzeigen würde.

\begin{figure}[H]
    \begin{center}\includegraphics[width=\textwidth]{foto.png}\end{center}
\caption{Fertig bestückte Platine.}
\label{fig:aufbau}
\end{figure}

\section{Stückliste}

\begin{center}\begin{tabular}{llll}\toprule
    \textbf{Referenz} & \textbf{Bauteilwert} & \textbf{Gehäuse} & \textbf{Beschriftung / Bemerkung}\\\midrule
Q1 & MMBF4393LT1G & SMD SOT-23 & M6G / N Kanal J-FET\\ % FIXME check marking
Q2 & MMBTA63LT1G  & SMD SOT-23 & 2U / PNP Darlington\\ 
D1 & BAT54S       & SMD SOT-23 & WV4 / Schottky Diode\\ %  FIXME check marking
R1 & 10~k$\Omega$ & SMD 1206 & 103 \\
R2 & 470~$\Omega$ & SMD 1206 & 471 \\
R3 & 1~M$\Omega$  & SMD 1206 & 105 \\
C1 & 100~pF       & SMD 1206 & (hellgrau) \\
L1 & 1~mH         & SMD 1210   & 102 (breit) \\
L3 & Ferritperle  & SMD 1206   & (schwarz) \\\midrule
D2 & LED (R oder B)    & 3~mm bedrahtet & Kathode ist das kurze Bein\\
P1 & 5~k$\Omega$  & bedrahtet  & 25 Umdrehungen\\\bottomrule
\end{tabular}\end{center}

Die Bestückungsreihenfolge ist grundsätzlich unkritisch, die obige Tabelle
dient als Vorschlag. Es bietet sich jedoch an, vor dem Einsetzen der bedrahteten
Bauteile (D2, P1, die vierpolige Stiftreihe und die beiden Atennendrähte)
alle SMD-Bauteile zu bestücken. So liegt die Platine für sämtliche SMD Bauteile
flach auf der Arbeitsfläche auf. Die SMD-Bauteile sind auf einer Referenzkarte
befestigt und dort auch beschrieben und farblich markiert. Der Batterieclip
wird an die Pins ``+'' und ``-'' der vierpoligen Stiftenreihe gelötet. Die
erwendung von Schrumpfschlauch zur mechanischen Stabilisierung dieses
Anschlusses dringend empfohlen. Polarität: (``+'' ist rot;``-'' ist schwarz).

\section{Justierung}
P1 wird wird zur Justierung langsam in Abwesenheit von HF- und Störsignalen im
Uhrzeigersinn gedreht, bis die LED ganz leicht zu leuchten beginnt. Vorsicht:
Bereits das Berühren der Platine oder Drähte kann HF-Energie einkoppeln!
Der so gefundene Arbeitspunkt kann durchaus fast am anderen Ende (also nach
beinahe 25 Umdrehungen) liegen und ist Abhängig von der Eingangsspannung. Deshalb
ist beim Entladen der Batterie eine periodische Korrektur notwendig. Das
Nachrüsten mit einem LDO (Spannungsregler) schafft hierfür Abhilfe.

\section{Montage in Gehäuse}
Die Platine ist zur Monage in ein RND 455-00848 Metallgehäuse ausgelegt. Dazu
werden in den Deckel 4 Bohrungen für M3 Senkschrauben gebohrt, sowie eine 3~mm
Bohrung für die LED und zwei 1~mm Bohrungen für die beiden Antennendrähte. Jede
dieser Schrauben wird mit einer M3 Mutter in den Deckel fixiert. Eine weitere
M3 Mutter presst dann die Platine gegen diese erste Mutter, so dass für die
vier Schrauben insgesamt acht Muttern benötigt werden.
Der Ein/Aus-Schalter und allenfalls ein 9~V DC Anschluss und das Potentiometer
können dann herausgeführt werden. Für eine steckbare Ausführung kann J1
mit einer JST XH Buchse bestückt werden. Bei der Montage durch den Deckel
werden die LED und die beiden Drähte auf der Platinenrückseite (also genau
anders als in Abbildung \ref{fig:aufbau} gezeigt!) montiert und durch den
Deckel nach aussen geführt.
\end{document}



