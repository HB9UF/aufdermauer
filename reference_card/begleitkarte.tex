\documentclass[12pt]{minimal}
\usepackage[papersize={85mm,70mm},hmargin=5mm,vmargin=5mm]{geometry}
\usepackage{booktabs}
\usepackage{wasysym}
\newcommand{\sanespace}{[1mm]}
\renewcommand{\arraystretch}{1.5}
\begin{document}
\begin{center}\begin{tabular}{lll}
\Square~~  & Q1~~ & Schwarzes Tape, keine Farbe \\
\Square & Q2 & Schwarzes Tape, weisse Farbe\\
\Square & D1 & Schwarzes Tape, orange Farbe\\
\Square & R1 & Weisses Tape, grüner Filzstift\\
\Square & R2 & Weisses Tape, roter Filzstift\\
\Square & R3 & Weisses Tape, blauer Filzstift\\
\Square & C1 & Weisses Tape, schwarzer Filzstift\\
\Square & L1 & Transparentes Tape (breites Bauteil)\\
\Square & L3 & Weisses Tape, kein Filzstift\\
\end{tabular}\end{center}
\end{document}
